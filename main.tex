\documentclass{article}
\usepackage[polish]{babel}
\usepackage[utf8]{inputenc}
\usepackage{polski}
\frenchspacing
\setcounter{tocdepth}{2}

\begin{document}
	
\begin{titlepage}

\newcommand{\HRule}{\rule{\linewidth}{0.5mm}}

\center

%----------------------------------------------------------------------------------------

\textsc{\LARGE Politechnika Warszawska}\\[5mm]
\textsc{\LARGE Wydział Matematyki i Nauk Informacyjnych}\\[4cm]
 
%----------------------------------------------------------------------------------------

\textsc{\Huge Algorytmy Zaawansowane}\\[0.5cm]

%----------------------------------------------------------------------------------------

\HRule \\[0.4cm]
{ \LARGE \bfseries Wyznaczanie spójności krawędziowej grafu przez przepływ}\\[0.2cm]
 
%----------------------------------------------------------------------------------------

\HRule \\[0.4cm]
{  \bfseries Dokumentacja wstępna projektu}\\[2.5cm]
 
%----------------------------------------------------------------------------------------

\begin{flushright}
\Large \emph{Autorzy:}\\[0.5cm]
Anna \textsc{Zawadzka}\\
Piotr \textsc{Waszkiewicz}\\
\end{flushright}
\\[3.5cm]
%----------------------------------------------------------------------------------------

\vfill
{\large \today}\\[3cm]

\end{titlepage}
	
\newpage

\section{Opis problemu}\\

Celem projektu jest zaprojektowanie i zaimplementowanie algorytmu znajdującego spójność krawędziową grafu przez przepływ.\\

Daną wejściową problemu jest nieskierowany graf bez wag $G=(V,E)$. Na jego podstawie utworzona będzie sieć przepływowa, czyli graf skierowany $G'=(V,E')$ z dodatnimi wagami określającymi przepustowości (pojemności) krawędzi i wyróżnionymi dwoma wierzchołkami: źródłem i ujściem. Następnie wyznaczony zostanie przepływ. Jest to fukcja $f$ określona na zbiorze $E'$ krawędzi grafu $G'$ taka, że:
\begin{itemize}
    \item dla każdego $e\in E'$ zachodzi $0\le f(e)\le waga(e)$
    \item dla każdego wierzchołka wewnętrznego (tzn. każdego oprócz źródła i ujścia) sumaryczny przepływ dopływający do tego wierzchołka jest równy sumarycznemu przepływowi wypływającemu z niego
\end{itemize}
Na podstawie przepływu możliwe będzie wyznaczenie spójności krawędziowej grafu wejściowego.

%----------------------------------------------------------------------------------------

\section{Metoda realizacji zadania}

Algorytm rozwiązania zadania jest następujący:
\begin{enumerate}
\item Na podstawie grafu wejściowego konstruujemy sieć przepływową, przy czym jednej krawędzi nieskierowanej grafu wejściowego odpowiadają dwie krawędzie skierowane w przeciwne strony w sieci przepływowej
\item Każdej krawędzi w sieci przepływowej nadajemy wagę o wartości 1
\item Dla każdej pary wierzchołków sieci przepływowej wyznaczamy maksymalny przepływ między nimi przy użyciu algorytmu Forda-Fulkersona
\item Ze wszystkich wyznaczonych maksymalnych przepływów wybieramy ten o minimalnej wartości
\item Określamy spójność krawędziową grafu wejściowego, która jest równa przepływowi wyznaczonemu w poprzednim kroku
\end{enumerate}

%----------------------------------------------------------------------------------------

\section{Analiza poprawności i złożoności czasowej algorytmu}

\subsection{Analiza poprawności}


Wyznaczenie spójności krawędziowej przy pomocy przepływu możliwe jest przy wykorzystaniu poniższego twierdzenia. \\ \\
\textbf{Twierdzenie o maksymalnym przepływie i minimalnym przekroju.}\\

\textit{Maksymalna wartość przepływu w sieci równa jest minimalnej przepustowości przekroju tej sieci.}\\\\
\textbf{Definicja.}\\

\textit{Przekrojem sieci przepływowej $G'=(V,E')$ nazywamy podział zbioru $V$ na zbiory $S$ i $T$ takie, że:
\begin{itemize}
    \item $S\cup T=V$ oraz $S\cap T=\emptyset$
    \item $s\in S$, gdzie $s$ - źródło
    \item $t\in T$, gdzie $t$ - ujście
\end{itemize}
Przepustowością przekroju nazywamy sumę wag wszystkich krawędzi o początku w $S$ i końcu w $T$.
}\\\\

Ponieważ w skonstruowanej sieci przepływowej wszystkie krawędzie mają przepustowości o wartości 1, maksymalny przepływ jest równy minimalnej liczbie krawędzi, których usunięcie spowoduje rozspójnienie grafu.\\


\subsection{Analiza złożoności czasowej}


Złożoności czasowe poszczególnych kroków algorytmu: 
\begin{enumerate}
\item Konstrukcja sieci przepływowej: $O(NIE WIEM)$
\item Ustalenie wag krawędzi sieci przepływowej: $O(|E'|)$
\item Wyznaczenie maksymalnego przepływu między wszystkimi parami wierzchołków: 
\begin{itemize}
    \item przejście po wszystkich parach: $O(\frac{|V|^{2}}{2})$
    \item obliczenie maksymalnego przepływu: $O(|V|\cdot |E'|\cdot |M|)$, gdzie $M$ - maksymalna pojemność krawędzi w grafie przepływowym
\end{itemize}
W tym rozwiązaniu wszystkie pojemności krawędzi grafu przepływowego mają wartość 1, zatem sumaryczna złożoność tego kroku to: $O(\frac{|V|^{2}}{2}\cdot|V|\cdot |E'|) \approx O(|V|^{3}\cdot |E'|)$
\item Wybór minimalnego przepływu spośród wyznaczonych w poprzednim kroku: $O(\frac{|V|^{2}}{2})$
\end{enumerate}\\
Zatem całkowita złożoność czasowa jest rzędu $O(|V|^{3}\cdot |E'|)$, gdzie $|E'|=2\cdot |E|$.\\


\section{Format danych wejściowych i wyjściowych}
Graf wejściowy będzie wprowadzany do programu w postaci pliku tekstowego, ale również będzie mógł być tworzony bezpośrednio w programie.\\

Format pliku tekstowego: pierwsza linia zawiera liczbę wierzchołków grafu $|V|$, każda kolejna linia reprezentuje krawędź grafu zdefiniowaną przez numery wierzchołków, będących końcami krawędzi, oddzielone spacją. Kolejność podawania numerów wierzchołków w definicji krawędzi nie ma znaczenia, gdyż graf wejściowy jest nieskierowany. Zakładamy, że wierzchołki grafu numerowane są od 0.
\\\\
Przykładowy plik wejściowy:\\
\textit{5\\
0 3\\
2 1\\
4 0\\
3 1\\
4 2\\
2 0\\
2 3\\
}


Wynikiem działania programu jest liczba określająca spójność krawędziową grafu wejściówego. Będzie ona widoczna bezpośrednnio w programie.

\end{document}



