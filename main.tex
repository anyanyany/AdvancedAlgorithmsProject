\documentclass{article}
\usepackage[polish]{babel}
\usepackage[utf8]{inputenc}
\usepackage{polski}
\frenchspacing
\setcounter{tocdepth}{2}

\begin{document}
	
\begin{titlepage}

\newcommand{\HRule}{\rule{\linewidth}{0.5mm}}

\center

%----------------------------------------------------------------------------------------

\textsc{\LARGE Politechnika Warszawska}\\[5mm]
\textsc{\LARGE Wydział Matematyki i Nauk Informacyjnych}\\[4cm]
 
%----------------------------------------------------------------------------------------

\textsc{\Huge Algorytmy Zaawansowane}\\[0.5cm]

%----------------------------------------------------------------------------------------

\HRule \\[0.4cm]
{ \LARGE \bfseries Wyznaczanie spójności krawędziowej grafu przez przepływ}\\[0.2cm]
 
%----------------------------------------------------------------------------------------

\HRule \\[0.4cm]
{  \bfseries Dokumentacja wstępna projektu}\\[2.5cm]
 
%----------------------------------------------------------------------------------------

\begin{flushright}
\Large \emph{Autorzy:}\\[0.5cm]
Anna \textsc{Zawadzka}\\
Piotr \textsc{Waszkiewicz}\\[3.5cm]
\end{flushright}
%----------------------------------------------------------------------------------------

\vfill
{\large \today}\\[3cm]

\end{titlepage}
	
\newpage

\section{Opis problemu}

Celem projektu jest zaprojektowanie i zaimplementowanie algorytmu znajdującego spójność krawędziową grafu poprzez wyznaczenie maksymalnego przepływu.\\

Daną wejściową problemu jest nieskierowany graf bez wag $G=(V,E)$. Na jego podstawie utworzona będzie sieć przepływowa, czyli graf skierowany $G'=(V,E')$ z dodatnimi wagami określającymi przepustowości (pojemności) krawędzi i wyróżnionymi dwoma wierzchołkami: źródłem i ujściem. Następnie wyznaczony zostanie przepływ. Jest to fukcja $f$ określona na zbiorze $E'$ krawędzi grafu $G'$ taka, że:
\begin{itemize}
    \item dla każdego $e\in E'$ zachodzi $0\le f(e)\le$ przepustowość$(e)$
    \item dla każdego wierzchołka wewnętrznego (tzn. każdego oprócz źródła i ujścia) sumaryczny przepływ dopływający do tego wierzchołka jest równy sumarycznemu przepływowi wypływającemu z niego
\end{itemize}
Na podstawie przepływu możliwe będzie wyznaczenie spójności krawędziowej grafu wejściowego.

%----------------------------------------------------------------------------------------

\section{Metoda realizacji zadania}

Algorytm rozwiązania zadania jest następujący:
\begin{enumerate}
\item Na podstawie grafu wejściowego konstruujemy sieć przepływową, przy czym jednej krawędzi nieskierowanej grafu wejściowego odpowiadają dwie krawędzie przeciwnie skierowane w sieci przepływowej
\item Wartości przepustowości każdej krawędzi w sieci przepływowej ustalamy na 1
\item Wybieramy jeden dowolny wierzchołek. Oznaczamy go jako źródło $s$
\item Dla każdej pary wierzchołków sieci przepływowej złożonej z wyróżnionego źródła $s$ i dowolnego innego wierzchołka $t$ ($t \neq s$) wyznaczamy maksymalny przepływ między nimi przy użyciu algorytmu Edmondsa-Karpa, będącego realizacją metody Forda-Fulkersona
\item Ze wszystkich wyznaczonych maksymalnych przepływów wybieramy ten o minimalnej wartości
\item Określamy spójność krawędziową grafu wejściowego, która jest równa przepływowi wyznaczonemu w poprzednim kroku
\end{enumerate}

%----------------------------------------------------------------------------------------

\section{Analiza poprawności i złożoności czasowej algorytmu}

\subsection{Analiza poprawności}


Wyznaczenie spójności krawędziowej przy pomocy przepływu możliwe jest przy wykorzystaniu poniższego twierdzenia. \\ \\
\textbf{Twierdzenie o maksymalnym przepływie i minimalnym przekroju.}\\

\textit{Maksymalna wartość przepływu w sieci równa jest minimalnej przepustowości przekroju tej sieci.}\\\\
\textbf{Definicja.}\\

\textit{Przekrojem sieci przepływowej $G'=(V,E')$ nazywamy podział zbioru $V$ na zbiory $S$ i $T$ takie, że:
\begin{itemize}
    \item $S\cup T=V$ oraz $S\cap T=\emptyset$
    \item $s\in S$, gdzie $s$ - źródło
    \item $t\in T$, gdzie $t$ - ujście
\end{itemize}
Przepustowością przekroju nazywamy sumę przepustowości wszystkich krawędzi o początku w $S$ i końcu w $T$.
}\\\\


Niech spójność krawędziowa grafu $G$ wynosi k, a wynik zwrócony przez algorytm to $f$ i niech $t$ będzie ujściem, dla którego mamy przepływ o wartości $f$.

Ponieważ  w  skonstruowanej  sieci  przepływowej  wszystkie  krawędzie  mają
przepustowości o wartości 1, to wartość minimalnej przepustowości
przekroju jest równa liczbie krawędzi w minimalnym przekroju. Usunięcie
tych krawędzi rozspójnia graf $G$, więc $k \le f$.

Ponieważ w przypadku rozspójnienia grafu k-spójnego w wyniku usunięcia k krawędzi zawsze otrzymujemy przynajmniej dwie wynikowe składowe, z których tylko jedna zawiera wyróżniony wcześniej wierzchołek źródłowy $s$, oraz nie posiada wierzchołka $t$ będącego ujściem, można zauważyć, że w przypadku dowolnego wyboru wierzchołka $s$ zawsze będzie istniał inny wierzchołek $t$ który w wyniku usunięcia krawędzi przynależał będzie do nowopowstałej składowej. Tak więc wystarczy dla wybranego źródła $s$ sprawdzić wszystkie możliwe kombinacje wierzchołków $t$ a następnie wybrać tę, która daje najmniejszą wartość przepustowości przekroju. 

Zatem istnieje zbiór $A$ k krawędzi których usunięcie rozspójnia graf. Wierzchołek $s$ jest w pewnej składowej $G-A$, więc dla dowolnego wierzchołka $t$ z innej składowej wartość minimalnego przekroju w sieci o źródle $s$ i ujściu $t$ będzie mniejszy lub równy $k$, więc $f \le k$.


\subsection{Analiza złożoności czasowej}


Złożoności czasowe poszczególnych kroków algorytmu: 
\begin{enumerate}
\item Konstrukcja sieci przepływowej: $O(|V|+|E|)$
\item Ustalenie wag krawędzi sieci przepływowej: $O(|E'|)$
\item Wybranie wierzchołka $s$: $O(1)$
\item Wyznaczenie maksymalnego przepływu między wybranymi parami wierzchołków: 
\begin{itemize}
    \item przejście po wszystkich wierzchołkach $t \neq s$: $O(|V|)$
    \item obliczenie maksymalnego przepływu za pomocą algorytmu Edmondsa-Karpa: $O(|V|\cdot|E'|^{2})$
\end{itemize}
Sumaryczna złożoność tego kroku to: $O(|V^{3}|\cdot|E'|)$
\item Wybór minimalnego przepływu spośród wyznaczonych w poprzednim kroku: $O(1)$
\end{enumerate}
Zatem całkowita złożoność czasowa jest rzędu $O(|V^{3}|\cdot|E'|)$, gdzie $|E'|=2\cdot |E|$.\\


\section{Format danych wejściowych i wyjściowych}
Graf wejściowy będzie wprowadzany do programu w postaci pliku tekstowego, ale również będzie mógł być tworzony bezpośrednio w programie.\\

Format pliku tekstowego: pierwsza linia zawiera liczbę wierzchołków grafu $|V|$ oraz liczbę krawędzi $|E|$ oddzielone spacją, każda kolejna linia reprezentuje krawędź grafu zdefiniowaną przez numery wierzchołków, będących końcami krawędzi, oddzielone spacją. Kolejność podawania numerów wierzchołków w definicji krawędzi nie ma znaczenia, gdyż graf wejściowy jest nieskierowany. Zakładamy, że wierzchołki grafu numerowane są od 0.
\\\\
Przykładowy plik wejściowy:\\
\textit{5 6\\
0 3\\
2 1\\
4 0\\
3 1\\
4 2\\
2 0\\
}


Wynikiem działania programu jest liczba określająca spójność krawędziową grafu wejściówego. Będzie ona widoczna bezpośrednio w programie.

\end{document}



