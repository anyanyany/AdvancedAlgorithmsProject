\documentclass{article}
\usepackage[polish]{babel}
\usepackage[utf8]{inputenc}
\usepackage{polski}
\frenchspacing
\setcounter{tocdepth}{2}

\begin{document}
	
\begin{titlepage}

\newcommand{\HRule}{\rule{\linewidth}{0.5mm}}

\center

%----------------------------------------------------------------------------------------

\textsc{\LARGE Politechnika Warszawska}\\[5mm]
\textsc{\LARGE Wydział Matematyki i Nauk Informacyjnych}\\[4cm]
 
%----------------------------------------------------------------------------------------

\textsc{\Huge Algorytmy Zaawansowane}\\[0.5cm]

%----------------------------------------------------------------------------------------

\HRule \\[0.4cm]
{ \LARGE \bfseries Dokumentacja wstępna projektu}\\[2.5cm]
 
%----------------------------------------------------------------------------------------

\begin{flushright}
\Large \emph{Autorzy:}\\[0.5cm]
Anna \textsc{Zawadzka}\\
Piotr \textsc{Waszkiewicz}\\
\end{flushright}
\\[4.5cm]
%----------------------------------------------------------------------------------------

\vfill
{\large \today}\\[3cm]

\end{titlepage}
	
\newpage

\section{Opis problemu}

Celem projektu jest zaprojektowanie i zaimplementowanie algorytmu znajdującego spójność krawędziową grafu przez przepływ.

Daną wejściową problemu jest nieskierowany graf bez wag G=(V,E). Na jego podstawie utworzona będzie sieć przepływowa, czyli graf skierowany G=(V,E), w którym każda krawędź (u,v) należąca do zbioru krawędzi E ma nieujemną przepustowość c(u,v) $>=$ 0, dodatkowo wyróżnione są dwa wierzchołki: źródło s i ujście t. Następnie wyznaczony zostanie przepływ, przy pomocy którego będzie można uzyskać informację o spójności krawędziowej grafu wejściowego.


\section{Metoda realizacji zadania}

\begin{enumerate}
\item Obliczamy najkrótsze ścieżki między wszystkimi parami wierzchołków
\item Wybieramy najdłuższą z najkrótszych ścieżek, która wyznacza źródło i ujście sieci przepływowej
\item Skierowanie krawędzi wyznaczamy za pomocą algorytmu BFS
\item Przepustowości wszystkich krawędzi w sieci przepływowej ustalamy na 1
\item Wyznaczamy przepływ maksymalny algorytmem Forda-Fulkersona
\item Określamy spójność krawędziową grafu wejściowego
\end{enumerate}


Wyznaczenie spójności krawędziowej możliwe jest przy wykorzystaniu z twierdzenia:
W dowolnej sieci wartość maksymalnego przepływu jest równa przepustowości minimalnego przekroju.

\section{Anaiza poprawności i złożoności czasowej algorytmu}

\section{Opis wejścia i wyjścia}

\end{document}
